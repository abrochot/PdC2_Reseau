\section{Découpage du projet}

\paragraph{} Ce projet de passage en ToIP du campus de l'INSA de Lyon étant très large, il sera nécessaire de le découper en sous-projets indépendants afin de traiter les diverses modifications qu'il apportera au réseau du campus.

\paragraph{} Un premier sous-projet consiste en intégration au coeur du réseau des équipements nécessaires à la ToIP. Le réseau actuel n'étant pour l'instant complètement analogique, de nombreux équipements et formations du personnel seront à acheter afin de pouvoir exploiter dans le futur ces nouveaux moyens.

\paragraph{} Comme nous l'avons vu dans l'analyse de l'existant, le réseau actuel ne permettrait pas d'assurer un service de ToIP en plus du service de gestion de données actuel. Un deuxième sous-projet sera par conséquent mis en place pour modifier l'infrastructure générale du réseau actuel afin qu'il satisfasse aux exigences données par la ToIP.

\paragraph{} Enfin, une dernière partie de ce projet consiste à modifier le matériel end-user présent dans tous les bâtiments du campus afin d'intégrer la nouvelle architecture dans les différents bâtiments et permettre aux utilisateurs de réellement utiliser la nouvelle architecture décrite dans ce dossier.