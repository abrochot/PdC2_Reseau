\section{Déploiement}

\subsection{Volumétrie}

\paragraph{} Cette sous-section décrit une estimation du volume des postes et
équipements qui sont couverts par cette étude.

\paragraph{} TODO tableau...

\subsection{Réseau}

\paragraph{} L'installation du réseau de VoIP implique de mettre en place une
stratégie de sécurité tenant compte de ses spécificités. Il nous semble
raisonnable de cloisonner le sous-réseau voix au niveau logique, et donc de
placer l'ensemble des interfaces de VoIP dans un sous-réseau virtuel (VLAN)
dédié, afin de dissocier le réseau de données du réseau voix.

\paragraph{} Ce VLAN sera un VLAN de niveau 2 ou plus (vlan par adresse MAC),
car des ordinateurs seront raccordés en "bout de ligne" à travers le pont fourni
sur les téléphones.

\paragraph{} Par ailleurs, afin de garantir ce cloisonnement, des règles de
pare-feu contrôleront l'accès au sous-réseau voix. Ces règles limiteront
l'accès au réseau aux protocoles de VoIP supportés (voir la section
\ref{logiciel}) et aux canaux de données qui peuvent être utilisés par les
terminaux pour se mettre à jour (des réglages fins peuvent être envisagés avec
les spécifications des constructeurs).

\paragraph{} TODO adressage ?

\subsection{Répartition physique}

\paragraph{} De nombreuses contraintes physiques, comme la longueur limite d'une
ligne (du switch à la prise de l'utilisateur) ou la nécessité de climatiser les
locaux accueillant les switchs \ac{POE}, nous devons réfléchir à une restructure
des locaux techniques.
