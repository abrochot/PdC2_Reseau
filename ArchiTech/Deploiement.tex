\section{Déploiement}

\subsection{Réseau}

\paragraph{} L'installation du réseau de VoIP implique de mettre en place une
stratégie de sécurité tenant compte de ses spécificités. Il nous semble
raisonnable de cloisonner le sous-réseau voix au niveau logique, et donc de
placer l'ensemble des interfaces de VoIP dans un sous-réseau virtuel (VLAN)
dédié, afin de dissocier le réseau de données du réseau voix.

\paragraph{} Ce VLAN sera un VLAN de niveau 2 ou plus. L'appartenance à un VLAN
sera déterminé par les ports de switchs et les téléphones seront identifiés par
le addresse MAC, pour des raisons de sécurité.

\paragraph{} Par ailleurs, afin de garantir ce cloisonnement, des règles de
pare-feu contrôleront l'accès au sous-réseau voix. Ces règles limiteront
l'accès au réseau aux protocoles de VoIP supportés (voir la section
\ref{logiciel}) et aux canaux de données qui peuvent être utilisés par les
terminaux pour se mettre à jour (des réglages fins peuvent être envisagés avec
les spécifications des constructeurs).

\paragraph{} D'après l'étude volumétrique (voir la section Volumétrie), nous avons
besoin de 2310 postes, soit autant d'adresses IP. Les téléphones seront donc
dans un VLAN adressé dans une plage de classe B.

\subsection{Répartition physique}

\paragraph{} De nombreuses contraintes physiques, comme la longueur limite d'une
ligne (du switch à la prise de l'utilisateur) ou la nécessité de climatiser les
locaux accueillant les switchs \ac{POE}, nous devons réfléchir à une
restructuration des locaux techniques.

\paragraph{} D'une manière générale, on peut considérer qu'on local technique
que nous avons pu placer judicieusement (c'est à dire, généralement, au centre
d'un bâtiment) permettra d'être un point de départ pour le câblage de trois
étages : celui du dessous, celui qui accueille le local et celui du dessus.

\subsection{Volumétrie}

\paragraph{} Cette sous-section décrit une estimation du volume des postes et
équipements qui sont couverts par cette étude.

\paragraph{} Nous considérons qu'il faut en moyenne 40 prises dans les étages
des bâtiments d'enseignement et recherche, et les bâtiments de services communs
(qu'on regroupera dans une catégorie A).
Le grand restaurant accueille également des bureaux, on considérera
qu'il nécessite 40 prises par bâtiment. Les résidences, les locaux sportifs et
autres résidences nécessitent en moyenne 10 prises au total (on les regroupe
    dans une catégorie B).

\paragraph{} Le tableau \ref{table_switchs} estime le nombre de switchs \ac{POE}
nécessaires, en considérant qu'un switch est équipé de 24 ports \ac{POE} et 24
ports neutres. Il n'est généralement pas nécessaire d'équiper toutes les prises
de \ac{POE}. Un switch par étage est donc suffisant.

\begin{table}[p]
  \caption{\label{table_switchs} Calcul du nombre de switchs nécessaires}
  \begin{tabular}{|l|l|c|m{4cm}|c|}
    \hline
    Catégorie & Nature & Nombre & Nombre de switch \ac{POE} (par bâtiment) & Nombre
    de switch total \\
    \hline
    B & Résidences  & 11 & 1 & 11 \\
    B & Sport       & 3  & 1 & 3  \\
    B & Restaurants & 1  & 1 & 1  \\
    \hline
    B & Total       & 15 &       & 15 \\
    \hline
      & Grand restaurant & 1 & 1 & 1 \\
    \hline
    A & 2 étages         & 9 & 2 & 18 \\
    A & 3 étages         & 9 & 3 & 27 \\
    A & plus de 3 étages & 2 & 4 & 8 \\
    \hline
    A & Total            & 20 &  & 53 \\
    \hline
    \hline
      & \bf{Grand total} & & & \bf{69} \\
    \hline
  \end{tabular}
\end{table}

\paragraph{} On compte un onduleur et une baie de brassage par local technique. Le tableau
\ref{table_locaux} estime le nombre de locaux techniques nécessaires.

\begin{table}[p]
  \caption{\label{table_locaux} Calcul du nombre de locaux techniques nécessaires}
  \begin{tabular}{|l|c|m{4cm}|c|}
    \hline
    Nature & Nombre & Nombre de locaux (par bâtiment) & Nombre de locaux total \\
    \hline
    Résidences  & 11 & 1 & 11 \\
    Sport       & 3  & 1 & 3 \\
    Restaurants & 1  & 1 & 1 \\
    \hline
    Total       & 15 &   & 15 \\
    \hline
    Grand restaurant & 1 & 1 & 1 \\
    \hline
    2 étages         & 9 & 1 & 9 \\
    3 étages         & 9 & 1 & 9 \\
    plus de 3 étages & 2 & 2 & 4 \\
    \hline
    Total            & 20 & & 22 \\
    \hline
    \hline
    \bf{Grand total} & & & \bf{38} \\
    \bf{Onduleurs} & & & \bf{38} \\
    \bf{Baies de brassage} & & & \bf{38} \\
    \hline
  \end{tabular}
\end{table}

\paragraph{} On considérera en moyenne une longueur de câble de 40m par prise.
Le \ref{table_cable} estime le nombre de prises, de téléphones et la longueur de
câble nécessaire.

\begin{table}[p]
  \caption{\label{table_cable} Calcul de la longueur de câble nécessaire}
  \begin{tabular}{|l|l|c|m{4cm}|c|}
    \hline
    Catégorie & Nature & Nombre & Nombre de prises (par bâtiment) & Nombre
    de prises total \\
    \hline
    B & Résidences & 11 & 10 & 110 \\
    B & Sport & 3 & 10 & 30 \\
    B & Restaurants & 1 & 10 & 10 \\
    \hline
    B & Total & 15 & & 150 \\
    \hline
      & Grand restaurant & 1 & 40 & 40 \\
    \hline
    A & 2 étages         & 9 & 80  & 720 \\
    A & 3 étages         & 9 & 120 & 1080 \\
    A & plus de 3 étages & 2 & 160 & 320 \\
    \hline
    A & Total            & 20 & & 2120 \\
    \hline
    \hline
      & \bf{Grand total} & & & \bf{2310} \\
      & \bf{Téléphones IP} & & & \bf{2310} \\
      & \bf{Longueur de câble} & & & \bf{92,4 km} \\
    \hline
  \end{tabular}
\end{table}

