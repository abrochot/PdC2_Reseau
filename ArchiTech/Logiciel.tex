\section{\label{logiciels}Logiciels utilisés}

\paragraph{} Cette section décrira les logiciels qui seront utilisés dans le
réseau ToIP dont nous décrivons l'architecture. Par logiciel, nous entendons
discuter des systèmes d'exploitation, protocoles, applications et composants
logiciels sur le serveur.

\paragraph{} Plusieurs critères de sélection on été retenus, les voici :
\begin{description}
	\item[Interopérabilité] Les outils retenus doivent être compatibles avec le
	réseau existant, qui est très hétérogène. Les logiciels doivent donc utiliser
	des standards ouverts tant que possible.
	\item[Popularité] La popularité d'une solution est un indicateur relativement
	fiable de sa pérennité, de la facilité d'accès à des informations et des
	compétences spécifiques pour l'utiliser.
	\item[Solution Open-Source] Nous pensons que les solutions Open-Source
	offrent généralement des bons niveau d'interopérabilité (ou à défaut, la
	possibilité de modifier ces produits) et permettent de réduire les coûts.
	Les solutions Open-Source sont par ailleurs moins souvent liées à du matériel
	spécifique.
	\item[Prix] Le prix des licences de logiciels est naturellement un facteur de
	décision important pour deux solutions offrant des caractéristiques similaires
	répondant à nos besoins.
	\item[Complexité de la solution] Une solution complexe est une solution
	difficile à déployer et à maintenir, et qui peut se révéler, à moyen
	terme, coûteuse ou difficile à rendre fiable.
\end{description}

\subsection{Protocoles et codecs}

\begin{description}
	\item[RTP et RTCP] Les protocoles \ac{RTP} et \ac{RTCP} sont des protocoles
	standards de communication en temps réel sur un réseau IP.
	\item[SIP] \ac{SIP} est un protocole standard ouvert couramment utilisé en
	téléphonie par IP. Il remplace progressivement le standard H.323, qui souffre
	de limitations techniques et de failles de sécurité connues. SIP permet
	également de gérer des sessions de visiophonie, de messagerie instantanée ou
	des communications en mode conférence (plus de deux interlocuteurs).
\end{description}

\subsection{Call Manager et passerelle IP}

\paragraph{} Nous envisageons d'utiliser la solution
Asterisk\footnote{\url{www.asterisk.org}}, qui répond très bien à nos critères
de sélection. C'est une solution Open-Source reconnue qui intègre un vaste panel
d'outil dédiés à la ToIP.

\paragraph{} Asterisk dispose de nombreuses fonctionnalités et d'un mécanisme
d'extensions très complet. Il est notamment compatible avec les protocoles
\ac{PRI} T0 et T1, pour se relier à une ligne analogique ou \ac{RNIS}. Cette
suite logicielle sera donc en mesure de supporter l'architecture de transition
et l'architecture cible sur un serveur disposant des interfaces matérielles
nécessaires : occuper le rôle d'interface avec le réseau opérateur d'une part,
et remplacer le PABX analogique pour les terminaux résiduels.

\paragraph{} Asterisk sera déployé sur une distribution GNU/Linux d'entreprise
(type Redhat).

\subsection{Terminaux}

\paragraph{} Les autres équipements utiliseront généralement leurs firmwares
d'origine. Cependant, certains terminaux téléphonique pourront être remplacés
par des logiciels (clients SIP) installées sur un ordinateur si l'utilisateur le
souhaite. Cette configuration n'est pas évaluée dans le cadre de cette étude et
n'est pas recommandée, notamment pour des raisons de sécurité.
