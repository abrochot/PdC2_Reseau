\section{Matériel retenu}

\paragraph{} Cette section discute du matériel qui supportera l'infrastructure
du réseau ToIP sur le campus de l'INSA.

\paragraph{} Les principaux critères de sélection sont les suivants :
\begin{description}
	\item[Interopérabilité] Les outils retenus doivent être compatibles avec le
	réseau existant, qui est très hétérogène. Le matériel doit donc pouvoir
	travailler avec des produits commercialisés par la concurrence et permettre,
	si besoin, d'accueillir des cartes et interfaces réseau supplémentaires.
	\item[Prix] Le prix du matériel (à l'achat et pour la maintenance) sera
  déterminant, mais sera surtout discuté dans le cadre d'appels d'offres.
	\item[Popularité] La popularité d'une gamme est un indicateur sur
  l'interopérabilité du matériel et donne une bonne garantie sur sa pérennité.
\end{description}

\subsection{Serveur}

\paragraph{} Nous optons pour un serveur garantissant une très haute
disponibilité et une fiabilité accrue. Le serveur sera équipée de technologies
matérielles conçues pour ces situations (RAID, fallbacks matériels, etc).

\paragraph{} Par ailleurs, ce serveur sera équipé d'interfaces réseau dédiées à
la connexion vers le réseau \ac{RNIS} ou analogique, telles que celles proposées
par la branche Hardware de la société Asterisk. Les cartes de la série Hx8 sont
des cartes hybrides disposant d'interfaces analogiques et numériques.

\paragraph{} Nous estimons le coût d'achat d'un tel serveur à 5000 \euro HT.

\subsection{Switch POE}

\paragraph{} Les switchs qui doivent supporter le trafic de ToIP doivent être
équipés de technologies \ac{POE}, qui permet d'alimenter les terminaux
téléphoniques et éviter une interruption de service en cas de coupure
d'électricité. Ces switchs doivent être manageables (switchs de niveau 3) pour
permettre de créer des VLAN dédiés à la ToIP.

\paragraph{} Un switch POE accueille généralement une part de ports ethernet
équipés et des ports standards. Compte-tenu des investissements à prévoir de la
part de la DSI, nous pouvons envisager d'opter pour des switchs équivalents à
ceux qui sont prévus, mais disposant de l'option POE.

\paragraph{} Il faut noter que des switchs POE nécessitent des locaux équipés de
climatisation adaptée et un onduleur à batteries pour rester disponibles en cas
de panne d'alimentation.

\subsection{Téléphones IP}

\paragraph{} Les téléphones IP qui seront déployés doivent supporter les
protocoles que nous souhaitons utiliser, et principalement \ac{SIP}. La plupart
des terminaux du marchés sont compatibles avec une telle technologie.

\paragraph{} Les téléphones doivent aussi être en mesure de raccorder un
ordinateur en "bout de ligne" pour éviter de devoir réinstaller des prises RJ45
dans les locaux.
