\documentclass[a4paper]{article}

\usepackage{hyperref}
\hypersetup{
colorlinks=false, % bool: Liens colorés
pdfborder={0 0 0} % Ne pas encadrer les liens
}
\usepackage[utf8]{inputenc}
\usepackage[francais]{babel}
\usepackage[top=2cm, bottom=2cm, left=2cm, right=2cm]{geometry}
\usepackage{graphicx}
\usepackage[final]{pdfpages}
\usepackage{rotating}
\usepackage{eurosym}
\usepackage{lscape}
\usepackage{float}
\usepackage{color}
\usepackage{colortbl}
\usepackage[printonlyused]{acronym}

% définir les commandes ici

% s'il y a beaucoup de commandes et de packages à inclure n'h&ésitez pas
% à mettre tout ça dans un fichier include.tex et l'inclure
% \input{include.tex}




\begin{document}
\title{PdC 2 : Architecture technique détaillée}
\author{Elisa ABIDH, Adrien BROCHOT, Martin RICHARD, Jetmir XHEMBULLA}

%------------------------------------- Page de titre
\maketitle
%\begin{titlepage}
%~

%\vfill
%\begin{Large}
%Septembre 2011
%\end{Large}
%\vfill
%\end{titlepage}
%----------------------------------------------------

%--------------------------------- Table des matières
\newpage
\tableofcontents
\newpage
%----------------------------------------------- Plan

\section{Modèles d'architecture}

\paragraph{} Le déploiement du réseau de téléphonie par IP sur le campus de
l'INSA ne peut être réalisé en rupture avec l'infrastructure analogique
existante à une telle échelle.
\paragraph{} La section qui suit décrira à un niveau conceptuel l'architecture
du réseau de téléphonie existant, de celui que nous proposons à terme, et de
celui que nous proposons durant la phase de transition.

\subsection{Architecture existante}
\paragraph{} L'architecture actuelle repose sur un PABX analogique chargé
d'effectuer la commutation des lignes et d'assurer la liaison vers le réseau
Numéris de l'opérateur.

\subsection{Architecture de transition}
\paragraph{} Todo

\subsection{Architecture cible}
\paragraph{} Todo


\newpage
\section{Logiciels utilisés}

\paragraph{} Cette section décrira les logiciels qui seront utilisés dans le
réseau ToIP dont nous décrivons l'architecture. Par logiciel, nous entendons
discuter des systèmes d'exploitation, protocoles, applications et composants
logiciels sur le serveur.

\paragraph{} Plusieurs critères de sélection on été retenus, les voici :
\begin{description}
	\item[Interopérabilité] Les outils retenus doivent être compatibles avec le
	réseau existant, qui est très hétérogène. Les logiciels doivent donc utiliser
	des standards ouverts tant que possible.
	\item[Popularité] La popularité d'une solution est un indicateur relativement
	fiable de sa pérennité, de la facilité d'accès à des informations et des
	compétences spécifiques pour l'utiliser.
	\item[Solution Open-Source] Nous pensons que les solutions Open-Source
	offrent généralement des bons niveau d'interopérabilité (ou à défaut, la
	possibilité de modifier ces produits) et permettent de réduire les coûts.
	Les solutions Open-Source sont par ailleurs moins souvent liées à du matériel
	spécifique.
	\item[Prix] Le prix des licences de logiciels est naturellement un facteur de
	décision important pour deux solutions offrant des caractéristiques similaires
	répondant à nos besoins.
	\item[Complexité de la solution] Une solution complexe est une solution
	difficile à déployer et à maintenir, et qui peut se révéler, à moyen
	terme, coûteuse ou difficile à rendre fiable.
\end{description}

\subsection{Protocoles et codecs}

\begin{description}
	\item[RTP et RTCP] Les protocoles \ac{RTP} et \ac{RTCP} sont des protocoles
	standards de communication en temps réel sur un réseau IP.
	\item[SIP] \ac{SIP} est un protocole standard ouvert couramment utilisé en
	téléphonie par IP. Il remplace progressivement le standard H.323, qui souffre
	de limitations techniques et de failles de sécurité connues. SIP permet
	également de gérer des sessions de visiophonie, de messagerie instantanée ou
	des communications en mode conférence (plus de deux interlocuteurs).
\end{description}

\subsection{Call Manager et passerelle IP}

\paragraph{} Nous envisageons d'utiliser la solution
Asterisk\footnote{\url{www.asterisk.org}}, qui répond très bien à nos critères
de sélection. C'est une solution Open-Source reconnue qui intègre un vaste panel
d'outil dédiés à la ToIP.

\paragraph{} Asterisk dispose de nombreuses fonctionnalités et d'un mécanisme
d'extensions très complet. Il est notamment compatible avec les protocoles
\ac{PRI} T0 et T1, pour se relier à une ligne analogique ou \ac{RNIS}. Cette
suite logicielle sera donc en mesure de supporter l'architecture de transition
et l'architecture cible sur un serveur disposant des interfaces matérielles
nécessaires.

\paragraph{} Asterisk sera déployé sur une distribution GNU/Linux d'entreprise
(type Redhat).

\subsection{Terminaux}

\paragraph{} Les autres équipements utiliseront généralement leurs firmwares
d'origine. Cependant, certains terminaux téléphonique pourront être remplacés
par des logiciels (clients SIP) installées sur un ordinateur si l'utilisateur le
souhaite. Cette configuration n'est pas évaluée dans le cadre de cette étude et
n'est pas recommandée, notamment pour des raisons de sécurité.

\newpage
\section{Matériel retenu}

\paragraph{} Cette section discute du matériel qui supportera l'infrastructure
du réseau ToIP sur le campus de l'INSA.

\paragraph{} Les principaux critères de sélection sont les suivants :
\begin{description}
	\item[Interopérabilité] Les outils retenus doivent être compatibles avec le
	réseau existant, qui est très hétérogène. Le matériel doit donc pouvoir
	travailler avec des produits commercialisés par la concurrence et permettre,
	si besoin, d'accueillir des cartes et interfaces réseau supplémentaires.
	\item[Prix] Le prix du matériel (à l'achat et pour la maintenance) sera
  déterminant, mais sera surtout discuté dans le cadre d'appels d'offres.
	\item[Popularité] La popularité d'une gamme est un indicateur sur
  l'interopérabilité du matériel et donne une bonne garantie sur sa pérennité.
\end{description}

\subsection{PABX}

\paragraph{} Le \ac{PABX} existant sera conservé lors de la phase de transition.
Nous veillerons à optimiser sa charge et nous assurer que les systèmes
d'alimentation de secours en place sont toujours pleinement opérationnels.

\subsection{Serveur}

\paragraph{} Nous optons pour un serveur garantissant une très haute
disponibilité et une fiabilité accrue. Le serveur sera équipée de technologies
matérielles conçues pour ces situations (RAID, fallbacks matériels, etc).

\paragraph{} Par ailleurs, ce serveur sera équipé d'interfaces réseau dédiées à
la connexion vers le réseau \ac{RNIS} ou analogique, telles que celles proposées
par la branche Hardware de la société Asterisk. Les cartes de la série Hx8 sont
des cartes hybrides disposant d'interfaces analogiques et numériques.

\paragraph{} Comme tout serveur critique, celui-ci est rattaché à un onduleur et
une alimentation de secours.

\paragraph{} Nous estimons le coût d'achat d'un tel serveur à 5000 \euro HT.

\subsection{Équipements coeur de réseau}

\paragraph{} Le matériel déployé au coeur du réseau sera généralement de la même
nature que celui déployé pour le reste du réseau IP du campus. D'une manière
générale, l'étude du matériel existant ou prévu à l'achat doit être reconsidéré
en tenant compte du surplus de trafic engendré par les postes de téléphonies par
IP envisagés dans le cadre de l'architecture cible.

\subsection{Switch POE}

\paragraph{} Les switchs qui doivent supporter le trafic de ToIP doivent être
équipés de technologies \ac{POE}, qui permet d'alimenter les terminaux
téléphoniques et éviter une interruption de service en cas de coupure
d'électricité. Ces switchs doivent être manageables (switchs de niveau 3) pour
permettre de créer des VLAN dédiés à la ToIP.

\paragraph{} Un switch POE accueille généralement une part de ports ethernet
équipés et des ports standards. Compte-tenu des investissements à prévoir de la
part de la DSI, nous pouvons envisager d'opter pour des switchs équivalents à
ceux qui sont prévus, mais disposant de l'option POE.

\paragraph{} Il faut noter que des switchs POE nécessitent des locaux équipés de
climatisation adaptée et un onduleur à batteries pour rester disponibles en cas
de panne d'alimentation.

\subsection{Téléphones IP}

\paragraph{} Les téléphones IP qui seront déployés doivent supporter les
protocoles que nous souhaitons utiliser, et principalement \ac{SIP}. La plupart
des terminaux du marchés sont compatibles avec une telle technologie.

\paragraph{} Les téléphones sont alimentés grâce à la téchnologie \ac{POE} et ne
nécessitent pas d'attention particulière pour garantir leur bon fonctionnement.

\newpage
\section{Acronymes utilisés dans ce document}

\begin{acronym}
  \acro{PABX}{Private Automatic Branch eXchange}
  \acro{POE}{Power Over Ethernet}
	\acro{PRI}{Primary Rate Interface}
  \acro{RNIS}{Réseau numérique à intégration de services}
	\acro{RTCP}{Real Time Control Protocol}
	\acro{RTP}{Real Time Protocol}
	\acro{SIP}{Session Initiation Protocol}
\end{acronym}

%\newpage
%\input{Reseau.tex}
%\newpage

\end{document}
