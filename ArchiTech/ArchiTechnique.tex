\documentclass[a4paper]{article}

\usepackage{hyperref}
\hypersetup{
colorlinks=false, % bool: Liens colorés
pdfborder={0 0 0} % Ne pas encadrer les liens
}
\usepackage[utf8]{inputenc}
\usepackage[francais]{babel}
\usepackage[top=2cm, bottom=2cm, left=2cm, right=2cm]{geometry}
\usepackage{graphicx}
\usepackage[final]{pdfpages}
\usepackage{rotating}
\usepackage{eurosym}
\usepackage{lscape}
\usepackage{float}
\usepackage{color}
\usepackage{colortbl}
\usepackage[printonlyused]{acronym}

% définir les commandes ici

% s'il y a beaucoup de commandes et de packages à inclure n'h&ésitez pas
% à mettre tout ça dans un fichier include.tex et l'inclure
% \input{include.tex}




\begin{document}
\title{PdC 2 : Architecture technique détaillée}
\author{Elisa ABIDH, Adrien BROCHOT, Martin RICHARD, Jetmir XHEMBULLA}

%------------------------------------- Page de titre
\maketitle
%\begin{titlepage}
%~

%\vfill
%\begin{Large}
%Septembre 2011
%\end{Large}
%\vfill
%\end{titlepage}
%----------------------------------------------------

%--------------------------------- Table des matières
\newpage
\tableofcontents
\newpage
%----------------------------------------------- Plan

\section{Modèles d'architecture}

\paragraph{} Le déploiement du réseau de téléphonie par IP sur le campus de
l'INSA ne peut être réalisé en rupture avec l'infrastructure analogique
existante à une telle échelle.
\paragraph{} La section qui suit décrira à un niveau conceptuel l'architecture
du réseau de téléphonie existant, de celui que nous proposons à terme, et de
celui que nous proposons durant la phase de transition.

\subsection{Architecture existante}
\paragraph{} L'architecture actuelle repose sur un PABX analogique chargé
d'effectuer la commutation des lignes et d'assurer la liaison vers le réseau
Numéris de l'opérateur.

\subsection{Architecture de transition}
\paragraph{} Todo

\subsection{Architecture cible}
\paragraph{} Todo


\newpage
\section{Logiciels utilisés}

\paragraph{} Cette section décrira les logiciels qui seront utilisés dans le
réseau ToIP dont nous décrivons l'architecture. Par logiciel, nous entendons
discuter des systèmes d'exploitation, protocoles, applications et composants
logiciels sur le serveur.

\paragraph{} Plusieurs critères de sélection on été retenus, les voici :
\begin{description}
	\item[Interopérabilité] Les outils retenus doivent être compatibles avec le
	réseau existant, qui est très hétérogène. Les logiciels doivent donc utiliser
	des standards ouverts tant que possible.
	\item[Popularité] La popularité d'une solution est un indicateur relativement
	fiable de sa pérennité, de la facilité d'accès à des informations et des
	compétences spécifiques pour l'utiliser.
	\item[Solution Open-Source] Nous pensons que les solutions Open-Source
	offrent généralement des bons niveau d'interopérabilité (ou à défaut, la
	possibilité de modifier ces produits) et permettent de réduire les coûts.
	Les solutions Open-Source sont par ailleurs moins souvent liées à du matériel
	spécifique.
	\item[Prix] Le prix des licences de logiciels est naturellement un facteur de
	décision important pour deux solutions offrant des caractéristiques similaires
	répondant à nos besoins.
	\item[Complexité de la solution] Une solution complexe est une solution
	difficile à déployer et à maintenir, et qui peut se révéler, à moyen
	terme, coûteuse ou difficile à rendre fiable.
\end{description}

\subsection{Protocoles et codecs}

\begin{description}
	\item[RTP et RTCP] Les protocoles \ac{RTP} et \ac{RTCP} sont des protocoles
	standards de communication en temps réel sur un réseau IP.
	\item[SIP] \ac{SIP} est un protocole standard ouvert couramment utilisé en
	téléphonie par IP. Il remplace progressivement le standard H.323, qui souffre
	de limitations techniques et de failles de sécurité connues. SIP permet
	également de gérer des sessions de visiophonie, de messagerie instantanée ou
	des communications en mode conférence (plus de deux interlocuteurs).
\end{description}

\subsection{Call Manager}

\paragraph{} Nous envisageons d'utiliser la solution
Asterisk\footnote{\url{www.asterisk.org}}, qui répond très bien à nos critères
de sélection.

\paragraph{} Asterisk dispose de nombreuses fonctionnalités et dispose d'un
mécanisme d'extensions très complet. Il est notamment compatible avec les
Protocoles \ac{PRI} T0 et T1, pour se relier à une ligne \ac{RNIS}.

\subsection{Passerelle IP}
\subsection{Switch POE}
\subsection{Téléphones IP}

\newpage
\section{Matériel retenu}

\paragraph{} Cette section discute du matériel qui supportera l'infrastructure
du réseau ToIP sur le campus de l'INSA.

\paragraph{} Les principaux critères de sélection sont les suivants :
\begin{description}
	\item[Interopérabilité] Les outils retenus doivent être compatibles avec le
	réseau existant, qui est très hétérogène. Le matériel doit donc pouvoir
	travailler avec des produits commercialisés par la concurrence et permettre,
	si besoin, d'accueillir des cartes et interfaces réseau supplémentaires.
	\item[Prix] Le prix du matériel (à l'achat et pour la maintenance) sera
  déterminant, mais sera surtout discuté dans le cadre d'appels d'offres.
	\item[Popularité] La popularité d'une gamme est un indicateur sur
  l'interopérabilité du matériel et donne une bonne garantie sur sa pérennité.
\end{description}

\subsection{PABX}

\paragraph{} Le \ac{PABX} existant sera conservé lors de la phase de transition.
Nous veillerons à optimiser sa charge et nous assurer que les systèmes
d'alimentation de secours en place sont toujours pleinement opérationnels.

\subsection{Serveur}

\paragraph{} Nous optons pour un serveur garantissant une très haute
disponibilité et une fiabilité accrue. Le serveur sera équipée de technologies
matérielles conçues pour ces situations (RAID, fallbacks matériels, etc).

\paragraph{} Par ailleurs, ce serveur sera équipé d'interfaces réseau dédiées à
la connexion vers le réseau \ac{RNIS} ou analogique, telles que celles proposées
par la branche Hardware de la société Asterisk. Les cartes de la série Hx8 sont
des cartes hybrides disposant d'interfaces analogiques et numériques.

\paragraph{} Comme tout serveur critique, celui-ci est rattaché à un onduleur et
une alimentation de secours.

\paragraph{} Nous estimons le coût d'achat d'un tel serveur à 5000 \euro HT.

\subsection{Équipements coeur de réseau}

\paragraph{} Le matériel déployé au coeur du réseau sera généralement de la même
nature que celui déployé pour le reste du réseau IP du campus. D'une manière
générale, l'étude du matériel existant ou prévu à l'achat doit être reconsidéré
en tenant compte du surplus de trafic engendré par les postes de téléphonies par
IP envisagés dans le cadre de l'architecture cible.

\subsection{Switch POE}

\paragraph{} Les switchs qui doivent supporter le trafic de ToIP doivent être
équipés de technologies \ac{POE}, qui permet d'alimenter les terminaux
téléphoniques et éviter une interruption de service en cas de coupure
d'électricité. Ces switchs doivent être manageables (switchs de niveau 3) pour
permettre de créer des VLAN dédiés à la ToIP.

\paragraph{} Un switch POE accueille généralement une part de ports ethernet
équipés et des ports standards. Compte-tenu des investissements à prévoir de la
part de la DSI, nous pouvons envisager d'opter pour des switchs équivalents à
ceux qui sont prévus, mais disposant de l'option POE.

\paragraph{} Il faut noter que des switchs POE nécessitent des locaux équipés de
climatisation adaptée et un onduleur à batteries pour rester disponibles en cas
de panne d'alimentation.

\subsection{Téléphones IP}

\paragraph{} Les téléphones IP qui seront déployés doivent supporter les
protocoles que nous souhaitons utiliser, et principalement \ac{SIP}. La plupart
des terminaux du marchés sont compatibles avec une telle technologie.

\paragraph{} Les téléphones doivent aussi être en mesure de raccorder un
ordinateur en "bout de ligne" pour éviter de devoir réinstaller des prises RJ45
dans les locaux.

\paragraph{} Les téléphones sont alimentés grâce à la téchnologie \ac{POE} et ne
nécessitent pas d'attention particulière pour garantir leur bon fonctionnement.

\newpage
\section{Définition des Boucles Optiques}

\paragraph{} Afin de déployer une nouvelle infrastructure réseau acceptant la ToIP dans le but de remplacer à terme l'ensemble des téléphones analogiques par des téléphones numériques (y compris les téléphones des secours), il est nécessaire de pouvoir assurer le transfert de données en toute circonstances. La sécurité du réseau doit être assurée par une double connexion de chaque bâtiment au réseau optique du campus. De cette manière, même en cas de défaillance d'une fibre, les données pourront toujours circuler. L'unique boucle optique actuellement disponible sur le campus ne répond pas aux exigences de sécurité citées ci-dessus. Elle est également déployée dans des fourreaux anciens et ne répondant pas aux normes de sécurité actuelles. Enfin, les fibres sont principalement des fibres optiques multimode ne permettant pas un débit suffisant pour gérer à la fois le réseau de données et le réseau téléphonique qui devront passer par les mêmes voies.

\paragraph{} De grands travaux d'aménagement seront effectués lors de la mise en place du projet de déploiement de ToIP sur le campus. Deux nouvelles connexions de fibres optiques seront mises en place : une boucle et une étoile. Ces connexion seront installées dans de nouveaux fourreaux enterrés dans tout le campus. Nous utiliserons des fibres monomodes, les nouvelles connexions formeront un réseau permettant à chaque bâtiment d'avoir deux accès indépendants au réseau global du campus. Les figures ci-dessous présentent les deux connexions avec les nœuds optiques associés.

\paragraph{} Les boucles seront créées selon un schéma de déploiement en étoile autour du bâtiment Jacquard qui représente le cœur du réseau. La boucle (qui est la connexion principale) est en plus conçue pour transporter les données d'un nœud à l'autre même en cas de section d'une partie de la boucle.


\begin{figure}[h]
  \caption{\label{Plan_boucle1} Première boucle optique}
  \includegraphics[scale=0.6]{Boucle1.png}
\end{figure}

\begin{figure}[h]
  \caption{\label{Plan_boucle2} Deuxième boucle optique}
  \includegraphics[scale=0.6]{Boucle2.png}
\end{figure}
~\\

\paragraph{} Nous pouvons remarquer que le déploiement de ces boucles conduira à la construction de 16 nœuds optiques et entrainera la pose de 6000 mètres de fourreaux pour la boucle 1 et 4500 mètres pour la seconde. Ces nœuds seront les points de convergence des fourreaux. L'alimentation électrique de ces nœuds se fera par câbles électriques enterrés dans des fourreaux parallèles aux fourreaux de fibre optique. Cette connexion permettra non seulement de n'avoir à effectuer qu'une seule tranchée pour les fibres et l'alimentation mais également de permettre de centraliser l'alimentation des différents nœuds et locaux techniques au bâtiment Jacquard ce qui permettra d'y installer les équipements de protection en cas de coupure de courant sur le campus (onduleur sur batterie, groupe électrogène).

\paragraph{} Deux chambres de tirage de fibre optique seront créées par bâtiment raccordé au boucles (une par connexion). Ces deux chambres de tirage raccorderont chaque bâtiment par deux emplacements distincts et espacés géographiquement afin de réduire les risques de déconnexion des bâtiments en cas de sinistre dans une partie de celui-ci.

\paragraph{} Comme présenté dans les plans, les fourreaux seront placés entre les bâtiments connectés. Afin de limiter le nombre de nœuds optiques, certaines fibres passeront à l'intérieur d'un bâtiment et seront redirigées vers un autre fourreau de sortie dans le local technique de ce dernier pour connecter un bâtiment voisin. Afin de prévenir de tout risque causé par le destruction de fourreaux aux abords des locaux techniques, chaque bâtiment connecté au réseau disposera de deux points d'entrées bien distincts pour les connexions des deux boucles optiques (en général : un point d'entrée pour la première boucle au niveau du local technique, une deuxième point d'entrée pour la seconde à l'autre bout du bâtiment. La fibre sera amenées au local par le câblage interne du bâtiment).

\paragraph{} La plupart des bâtiments du campus sont déjà équipés de locaux techniques pour la gestion de leur parc informatique. Lors de la mise en place des fourreaux pour la connexion d'un bâtiment, ces locaux techniques seront rénovés et équipés avec le matériel nécessaire à la gestion des deux nouvelles connexions. La rénovation passera par le changement des switchs qui seront utilisés par les téléphones en switchs POE, l'installation de climatisation dans les locaux (les switchs POE émettent plus de chaleur en fonctionnement que les switchs normaux), la création des deux nouvelles chambres de tirage décrites ci-dessus.

\paragraph{} Si la place disponible dans les locaux techniques d'un bâtiment n'est pas suffisante, de nouveaux locaux pourront être créés après destruction des anciens. La politique sera alors la création d'un ou deux locaux techniques par bâtiment (en fonction de la taille de ce dernier). Ce locaux techniques seront situés le plus possible aux étages médians des bâtiments afin de limiter la longueur des câbles Ethernet dans les bâtiments. 
\newpage
\section{Acronymes utilisés dans ce document}

\begin{acronym}
  \acro{RNIS}{Réseau numérique à intégration de services}
	\acro{RTP}{Real Time Protocol}
	\acro{RTCP}{Real Time Control Protocol}
	\acro{SIP}{Session Initiation Protocol}
	\acro{PRI}{Primary Rate Interface}
\end{acronym}

%\newpage
%\input{Reseau.tex}
%\newpage

\end{document}
