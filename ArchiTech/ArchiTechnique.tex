\documentclass[a4paper]{article}

\usepackage{hyperref}
\hypersetup{
colorlinks=false, % bool: Liens colorés
pdfborder={0 0 0} % Ne pas encadrer les liens
}
\usepackage[utf8]{inputenc}
\usepackage[francais]{babel}
\usepackage[top=2cm, bottom=2cm, left=2cm, right=2cm]{geometry}
\usepackage{graphicx}
\usepackage[final]{pdfpages}
\usepackage{rotating}
\usepackage{eurosym}
\usepackage{lscape}
\usepackage{float}
\usepackage{color}
\usepackage{colortbl}
% définir les commandes ici

% s'il y a beaucoup de commandes et de packages à inclure n'h&ésitez pas
% à mettre tout ça dans un fichier include.tex et l'inclure
% \input{include.tex}




\begin{document}
\title{PdC 2 : Architecture technique détaillée}
\author{Elisa ABIDH, Adrien BROCHOT, Martin RICHARD, Jetmir XHEMBULLA}

%------------------------------------- Page de titre
\maketitle
%\begin{titlepage}
%~

%\vfill
%\begin{Large}
%Septembre 2011
%\end{Large}
%\vfill
%\end{titlepage}
%----------------------------------------------------

%--------------------------------- Table des matières
\newpage
\tableofcontents
\newpage
%----------------------------------------------- Plan

\section{Modèles d'architecture}

\paragraph{} Le déploiement du réseau de téléphonie par IP sur le campus de
l'INSA ne peut être réalisé en rupture avec l'infrastructure analogique
existante à une telle échelle.
\paragraph{} La section qui suit décrira à un niveau conceptuel l'architecture
du réseau de téléphonie existant, de celui que nous proposons à terme, et de
celui que nous proposons durant la phase de transition.

\subsection{Architecture existante}
\paragraph{} L'architecture actuelle repose sur un PABX analogique chargé
d'effectuer la commutation des lignes et d'assurer la liaison vers le réseau
Numéris de l'opérateur.

\subsection{Architecture de transition}
\paragraph{} Todo

\subsection{Architecture cible}
\paragraph{} Todo


%\newpage

\end{document}
