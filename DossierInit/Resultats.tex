\section{Résultats Attendus}
\paragraph{} Ce projet doit aborder de nombreux aspects lors de l'étude des solutions disponibles. Il doit s'adapter aux différents publics à qui le retour de l'étude sera fait et qui devront décider du déploiement futur de la nouvelle architecture du réseau du campus. Afin d'aborder tous les aspects de manière complète sans pour autant contraindre certains acteurs du futur projet à parcourir des documents très techniques et ne les concernant pas, plusieurs dossiers différents seront rendus, chacun ayant un objectif et un destinataire précis. \\


\subsection{Rapport d'aide décisionelle:} Rapport destiné à un public non technique, il présente la solution retenue lors  des études préliminaires. Il est destiné au client : la direction de l'INSA. Il offre une présentation complète du projet d'amménagement ainsi que la nouvelle architecture (technique, organisationnelle). Une présentation de l'aspect économique et financier de la solution sera également présente dans ce rapport. Il permettra de définir les hypothèses effectuées lors de l'étude sur les moyens à disposition, les caractéristiques de la solution proposée ainsi que les conclusions du groupe d'étude.

\subsection{Dossier d'étude architecturale:} Dossier technique, il définit les technologies retenues lors de l'étude des solutions possibles. Il décrit précisément l'architecture technique du nouveau réseau mis en place sur le campus. Il permet de visualiser le réseau tel qu'il sera au terme du déploiement suivant la solution retenue. Il contiendra de nombreux schémas présentant la nouvelle architecture, la nouvelle organisation générale du campus suite au changement d'infrastructure réseau, les plans de migration de certaines technologies désormais obsolètes et la présentation des éléments les remplaçant. Il ne se limitera pas à l'aspect physique, il présentera également les différents protocoles, et standards utilisés pour l'exploitation de la nouvelle architecture. Il dressera également un inventaire du matériel nécessaire à la mise en place de la nouvelle architecture décrite.

\subsection{Rapport d'organisation du déploiement:} Ce rapport décrit l'ensemble du cycle de vie du projet de déploiement depuis son lancement jusqu'à la fin des travaux. Il permet de donner au client  une idée du déroulement des diverses opérations effectuées sur le campus. Ce document est essentiel car il permet de justifier les choix effectués dans la solution de réorganisation du réseau choisie. Il constitue également une étude du mode de déploiement optimal trouvé par notre groupe d'étude pour mettre en place la nouvelle infrastructure réseau du campus en gênant un minimum les utilisateurs. Il permet de définir les rôles de chaque structure organisationnelle dans le projet de déploiement. Il décrira également un nouveau plan de gouvernance du réseau suite à ce projet afin de faciliter les interactions entre les différentes structures du campus.

\subsection{Document de cadrage budgetaire:} Lors de ce projet, de nombreux éléments matériels devront être achetés et installés à divers emplacements du campus. En fonction de l'ordonnancement définit par le rapport d'organisation du déploiement, il convient d'optimiser l'achat de ces équipements afin d'éviter d'avoir à gérer un stock important. Le but étant ici d'éviter l'achat d'équipements trop longtemps avant leur utilisation dans le déploiement du système mais également d'éviter de bloquer le déploiement à cause d'un manque d'équipement. Ce document décrit exhaustivement les tarifs des différents éléments nécessaires à la mise en place de la nouvelle architecture retenue lors de l'étude précédente. Il permet au client de planifier les différentes commandes partielles qu'il aura à effectuer lors du déploiement et de connaître le budget global de la solution proposée. Il dresse un plan de financement tout au long du projet et permet de chiffre chaque lot décrit dans le dossier de déploiement. Il ne se contente pas de chiffrer le matériel nécessaire à chaque opération mais également de prévoir les prix des travaux et des personnes travaillant sur ce projet. Il propose enfin un plan de financement afin de prévoir et diminuer le budget global de ce projet.
