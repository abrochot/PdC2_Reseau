\section{Objet du projet - Contexte}

\subsection{Contexte général du projet}

\paragraph{} Le réseau actuel du campus de l'INSA est actuellement limité et en partie vétuste. Notre client, la direction de l'INSA, a donc mis en place un plan de rénovation de ce dernier. Ce projet se place dans ce contexte mais vise avant tout à modifier l'architecture actuelle du réseau afin d'y intégrer la ToIP.
\paragraph{} L'étude portera sur l'intégralité du site principal sur lequel sont déployés les équipements réseaux. L'un des points particuliers de l'infrastructure à définir est la diversité des utilisateurs présents sur le campus. Ce dernier est en effet composé aussi bien de laboratoires de recherches que de résidences, de bibliothèque ou encore de bureaux pour services administratifs de l'INSA. Cette diversité rend l'infrastructure plus compliqués du fait de la multiplicité des usages qui lui sont demandés.
\paragraph{} Sur le campus de l'INSA, les réseaux données et voix sont, à l'heure actuelle, séparés. Cette séparation entraine un problème de gestion évident des deux systèmes de communication entrainant un manque de sécurité et d'efficacité global de l'infrastructure. Leur gestion est actuellement assurée par la direction informatique mais pour le cas du réseau téléphonique, les abonés disposent d'une ligne téléphonique classique et leurs appels sont rétro-facturés.
\paragraph{} L'ajout de fonctions ToIP dans le réseau du campus a pour but une homogénisation des ressources. La gestion des différents réseaux sera alors facilitée et une meilleur maîtrise des coûts de téléphonie. Cela permettra également, lors du rennouvellement des PABX de les passer également en ToIP.

\subsection{Objectifs du projet}
\paragraph{} La mise en place de la téléphonie par IP sur le campus de l'INSA demande une modification globale de l'architecture réseau des équipements du campus. Ce projet a pour but de dégager une solution pour cette migration. L'objectif étant ici d'effectuer un travail de consultant aboutissant à un appel d'offre précisant l'organisation et le déploiement des nouveaux équipement nécessaires.

\paragraph{} Ce projet comporte de multiples objectifs visibles au travers des différents documents et rapports de retour. Il devra permettre à la direction de l'INSA d'appréhender les modifications et les travaux à effectuer sur l'infrastructure réseau du campus. Il doit leur permettre de prendre une décision finale quand au lancement éventuel de l'appel d'offre sur le déploiement qui aura été décrit dans l'étude fournie.

\paragraph{} Ce projet a finalement pour objectif la définition d'une solution complète de réorganisation du réseau du campus de l'INSA permettant d'ajouter des possibilités de ToIP sur le campus. Il devra montrer des solutions techniques et définir une architecture plus fiable, plus sécurisée et plus robuste que celle actuellement utilisée. Il présentera également les budgets nécessaires au déploiement de cette nouvelle architecture ainsi que l'odonnancement des tâches de déploiement à effectuer sur les différentes installations du campus. Il est en effet impossible de négliger le temps de déploiement d'une telle migration lors de l'étude d'une solution.

