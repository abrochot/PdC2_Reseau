\section{Analyse des risques}
Afin de pouvoir anticiper la présence d'incidents et d'éviter qu'ils ne retardent le rendu des divers document, une brève analyse des risques a permi de définir les principaux facteurs de retard et de prévoir les actions à mener pour éviter de pénaliser l'équipe.
\\
\\~
\\
\begin{tabular}{|l|l|}
\hline Risque & Erreur de plannification \\ \hline
Description & En raison du manque d'expérience du chef de projet, il est \\ 
& probable que les temps estimés pour les différentes tâches de ce projet \\
& soient mal cotés. Les erreurs pourront peut-être se compenser mais il est \\ 
& possible que le retard de certaines tâches retardent le déroulement du \\ 
& planning. \\ \hline
Probabilité & Forte \\ \hline
Prévention & Le chef de projet contrôlera très précisément l'avancement en \\
& particulier lors du début du projet. Il essayera ensuite de modifier les \\ & temps pour leur donner des valeurs plus adaptées. \\ \hline
\end{tabular}
\\
\\~
\\ 
\begin{tabular}{|l|l|}
\hline Risque & Erreur de communication \\ \hline
Description & Il est possible que différents membres de l'équipe travaillant \\
& sur une même tâche (ou sur des tâches différentes) aboutissent à des \\
& conclusion et des choix de solution différentes. S'ils ne communiquenet \\
& pas convenablement leurs résultats, les livrables pourraient avoir \\
& des données contradictoires \\ \hline
Probabilité & Faible \\ \hline
Prévention & Le chef de projet encourragera les communications entre les \\
& membres de l'équipe et animera les réunnions de projet en demandant à \\
& chaque ressource ses résultats pour les tâches en cours afin que chacun \\
& puisse avoir une vision globale de l'avancement du projet \\ \hline 
\end{tabular}
\\
\\~
\\ 
\begin{tabular}{|l|l|}
\hline Risque & Indisponibilité d'une ressource \\ \hline
Description & Un membre de l'équipe peut être momentanément indisponible pour \\
& des raisons diverses dont la maladie, Ce projet doit cependant \\
& respecter les délais, l'équipe complète ne doit par conséquent  ne pas \\
& être bloquée. \\ \hline
Probabilité & Faible \\ \hline
Prévention & Le chef de projet se réservera le droit de modifier les équipes \\
& techniques en cas d'indisponibilité imprévue d'une ressource. \\ \hline 
\end{tabular}
