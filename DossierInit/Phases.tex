\section{Méthodes utilisées et phasage du projet}

\subsection{Méthodes adoptées}
\subsubsection{Ressources}
Pour ce projet, notre groupe d'étude comporte quatre personnes : Adrien BROCHOT, chef de projet, et Elisa ABIDH, Martin RICHARD, Jetmir XHEMBULLA qui constitueront l'équipe technique. A cause du faible effectif de l'équipe, aucun rôle de responsable qualité n'est officiellement attribué. Le chef de projet contrôlera cependant le respect des démarches qualités lors du rendu de documents.

\subsubsection{Outils utilisés}
\paragraph*{• Le mail:}Outil de communication essentiel dans le travail en équipe à distance, le mail sera principalement utilisé à des fins de planification de réunions lorsque celles ci n'auront pas été définies lors des périodes de travail au département. Le rendu de ce projet s'effectuera le 16 novembre 2012 sous la forme d'un mail envoyé au client.

\paragraph*{• Le dépôt git:}Afin de mettre en commun les parties/documents rédigés, un dépôt git a été créé et servira de centre de données. Tous les membres de l'équipe y auront accès et devront à chaque modification d'un document le mettre à jour sur le dépôt.

\paragraph*{• LaTeX:} Lors de ce projet, des documents seront écrits en LaTeX pour être générés en format pdf, simplement imprimable et parfaitement adaptés à l'envoi de rapports par mail.

\subsection{Description des phases}

\subsubsection{Phase 1: Étude préparatoire} Cette phase a pour objectif de donner à notre groupe d'étude une visibilité approfondie de l'infrastructure du réseau du campus, ainsi que de la solution déployée sur le campus de l'université Lyon 1 qui a subi une migration similaire à celle que nous nous apprêtons à planifier. 

\subsubsection{Phase 2: Définition de la solution} La première étape de ce projet consistera en la création de l'architecture que nous détaillerons plus tard dans les différents livrables. il conviendra d'effectuer l'étude complète de cette solution.

\subsubsection{Phase 3: Élaboration du plan de déploiement} La mise en place de la solution décrite dans la définition de la solution demande une organisation particulière. Il convient de présenter les moyens mis en œuvre pour le déploiement de cette solution. Cette phase aura pour but l'ordonnancement du plan de déploiement de la nouvelle infrastructure réseau sur le campus de l'INSA ainsi que de créer les différents livrables nécessaires : dossier de déploiement, bordereau de cadrage budgétaire. Cette phase consistera en l'ordonnancement

\subsubsection{Phase 4: Mise en place de la politique budgétaire} Lors de la définition de la solution, les principaux équipements ont été définis, cette phase vise donc à fournir le budget précis nécessaire au déploiement de la nouvelle infrastructure du réseau. En complément des tarifs exacts et des fournisseurs des différents équipements, une organisation des factures partielles et de la chronologie de paiement sera intégrée au bordereau de cadrage budgétaire, objectif de cette phase.
