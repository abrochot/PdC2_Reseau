\section{Méthodes utilisées et phasage du projet}

\subsection{Méthodes adoptées}
\subsubsection{Ressources}
Pour ce projet, notre groupe d'étude comporte quatre personnes : Adrien BROCHOT, chef de projet, et Elisa ABIDH, Martin RICHARD, Jetmir XHEMBULLA qui constitueront l'équipe technique. A cause du faible effectif de l'équipe, aucun rôle de chef de projet n'est officielement attribué. Le chef de projet contrôlera cependant le respect des démarches qualités lors du rendu de documents.

\subsubsection{Outils utilisés}
\paragraph*{• Le mail:}Outil de communication essentiel dans le travil en équipe à distance, le mail sera principalement utilisé à des fins de plannification de réunions lorsque celles ci n'auront pas été définies lors des périodes de travail au département.

\paragraph*{• Le dépot git:}Afin de mettre en commun les parties/documents rédigés, un dépot git a été créé et servira de centre de données. Tous les membres de l'équipe y auront accès et devront à chaque modification d'un document le mettre à jour sur le dépot.

\paragraph*{• LaTeX:}Lors de ce projet, tous les documents seront écrits en LaTeX

\subsection{Description des phases}

\subsubsection{1: Etude préparatoire} Cette phase a pour objectif de donner à notre groupe d'étude une visibilité approfondie de l'infrastructure du réseau du campus, ainsi que de la solution déployée sur le campus de l'université Lyon 1 qui a subi une migration similaire à celle que nous nous apprettons à plannifier. 

\subsubsection{Phase 2: Analyse décisionnelle} Suite à l'étude préparatoire, nous pouvons commencer à étudier un maximum de solution possibles lors de cette phase. Le but est ici d'envisager et de comparer plusieurs infrastructures. Leur comparaison n'étant pas encore effectuée, il convient de n'oublier aucune technologie possible et de lister un maximum d'équipements adaptés pour avoir un plus large choix lors de la sélection de la solution qui sera par la suite proposée au client. Lors de cette phase, il faut ensuite comparer et organiser les solutions trouvées lors du benchmarking. La solution la plus adaptée sera alors sélectionnée en fonction de divers critères.

\subsubsection{Phase 3: Définition de la solution} Une fois la solution la plus adaptée sélectionnées, il conviendra d'effectuer l'étude complète de cette solution dans le but de produire le rapport technique. Ce dernier correspond à l'étude approfondie et technique de la solution présentée dans le rapport décisionnel.    

\subsubsection{Phase 4: Elaboration du plan de déploiement} La mise en place de la solution choisie lors de l'analyse décisionnelle demande une organisation particulière. Après la rédaction du rapport décisionnel pour la direction générale, il convient de présenter les moyens mis en oeuvre pour le déploiement de cette solution. Cette phase aura pour but l'ordonnancement du plan de déploiement de la nouvelle infrastructure réseau sur le campus de l'INSA. 

\subsubsection{Phase 5: Mise en place de la politique budgétaire} Lors de la définition de la solution, les principaux équipements ont été défnis, cette phase vise donc à fournir le budget précis nécessaire au déploiement de la nouvelle infrastructure du réseau. En complément des tarifs exacts et des fournisseurs des différents équipements, une organisation des factures partielles et de la chronologie de paiement sera intégrée au bordereau de cadrage budgétaire, objectif de cette phase.
